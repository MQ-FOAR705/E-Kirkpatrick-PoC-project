\documentclass{article}
\usepackage[utf8]{inputenc}

\title{Proof of Concept}
\author{Ellen Kirkpatrick }
\date{September 2019}

%This sets the indent length of a paragraph 
\setlength{\parindent}{0em}

%This sets the whitespace between paragraphs
\setlength{\parskip}{1em}

\begin{document}

\maketitle

\section{Aim}
Further to my scoping and elaboration, I will focus on the tools that can be used to identify common themes between multiple sources from different databases and how they can be stored in the same place with their metadata, so that additional annotations can be added and they can have shared labels.

\section{User Stories}
\textbf{Note:} User stories must be completed in order listed. Each user story depends on the story before being completed. 
\subsection{Identify commonalities}
As a research student, I would like a program which can compare multiple sources at once to identify common themes and terms. This will help determine relevance of source to project.
\subsection{Extract information}
As a research student, I would like a storage program that can extract source and metadata from an online platform so I can create references and citations more efficiently.
\subsection{Store sources}
As a research student, I would like a program that can store multiple sources with metadata so I can access all research material in the same place. 
\subsection{Annotations}
As a research student, I would like to be able to add annotations to stored sources so they can be referred to more easily in the future.
\subsection{Grouping}
As a research student, I would like to link specific sources through tags or labels so they can be grouped together. 
\subsection{Export metadata}
As a research student, I would like to export metadata to a word processing program, such as Microsoft Word, in order to automate the reference list. 

\section{Acceptance Criteria}
\subsection{Identify commonalities}
As a research student, I should be able to:
\begin{enumerate}
    \item Find peer-reviewed journal article sources on multiple databases.
    \item Download sources to computer.
    \item Upload sources to Voyant.
    \item Use the trends and cirrus tools on Voyant to identify key themes and terms.
    \item Use reader tool to read abstracts and determine relevance.
\end{enumerate}
\subsection{Extract information}
As a research student, I should be able to:
\begin{enumerate}
    \item Open Zotero application on computer.
    \item Use contexts tool on Voyant to connect to Zotero.
    \item Extract source and metadata from Voyant onto Zotero for storage.
   \end{enumerate}
\subsection{Store sources}
As a research student, I should be able to:
\begin{enumerate}
    \item Check that the extracted source appears in Zotero.
    \item Click on each source to ensure metadata is available.
    \item Click on each source to make sure the original website or pdf can be accessed.
    \item Flag source if metadata or part of metadata is missing, or if the source is not available.
\end{enumerate}
\subsection{Annotations}
As a research student, I should be able to:
\begin{enumerate}
    \item Add annotations to sources in Zotero.
    \item Access annotations in future, and add to them more.
    \end{enumerate}
\subsection{Grouping}
As a research student, I should be able to:
\begin{enumerate}
    \item Add tags or labels to source according to relevance or specific theme.
    \item Search for specific groups of sources by these tags.
\end{enumerate}
\subsection{Export metadata}
As a research student, I should be able to:
\begin{enumerate}
    \item Export source metadata from Zotero to Microsoft Word.
    \item Check that metadata is correct.
    \item Add to bibliography list if required.
\end{enumerate}

\subsection{Pre-requisites:}
\begin{itemize}
    \item All user stories must be completed in order for full efficiency. 
    \item Voyant must have been tested prior with known sources to determine whether this is reliable. \textit{This was completed in elaboration.}
    \item Zotero must be installed on computer and a connecting icon added to the browser of choice (Mozilla Firefox) before beginning. \textit{This was completed in elaboration.}
    \end{itemize}


\end{document}
