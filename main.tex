\documentclass{article}
\usepackage[utf8]{inputenc}

\title{Proof of Concept}
\author{Ellen Kirkpatrick }
\date{September 2019}

%This sets the indent length of a paragraph 
\setlength{\parindent}{0em}

%This sets the whitespace between paragraphs
\setlength{\parskip}{1em}

\begin{document}

\maketitle

\section{Aim}
Further to my scoping and elaboration, I will focus on the tools that can be used to identify common themes between multiple sources from different databases and how they can be stored in the same place with their metadata, so that additional annotations can be added and they can have shared labels.

\section{User Stories}
\subsection{Identify commonalities}
As a research student, I would like a program which can compare multiple sources at once to identify common themes and terms. This will help determine relevance of source to project.
\subsection{Extract information}
As a research student, I would like a storage program that can extract source and metadata from an online platform so I can create references and citations more efficiently.
\subsection{Store sources}
As a research student, I would like a program that can store multiple sources with metadata so I can access all research material in the same place. 
\subsection{Annotations}
As a research student, I would like to be able to add annotations to stored sources so they can be referred to more easily in the future.
\subsection{Grouping}
As a research student, I would like to link specific sources through tags or labels so they can be grouped together. 
\subsection{Export metadata}
As a research student, I would like to export metadata to a word processing program, such as Microsoft Word, in order to automate the reference list. 
\section{Notes on User Stories}
Each user story has been broken into specific categories which represent their overall task. The acceptance criteria in the section below, is aligned with each of these categories. \\
Each user story should be completed in the listed order as for optimum outcomes, each story depends on the success of the one before. \\
For this project to work, user story "Identify Commonalities" and "Store Sources" must work. These two user stories are core to the project and represent the overall aims of being able to compare multiple sources at once and store them together. "Extracting information" and "Export metadata" user stories are also crucial for the functional aspects of this project. \\
"Annotations" and "Grouping" are less essential user stories. Although, they are very useful for specificying sources to the researchers needs. Yet, sources can be compared and stored adequately without these user stories.

\section{Acceptance Criteria}
\subsection{Identify commonalities}
As a research student, I should be able to:
\begin{enumerate}
    \item Find peer-reviewed journal article sources on multiple databases.
    \item Download sources to computer.
    \item Upload sources to Voyant.
    \item Use the trends and cirrus tools on Voyant to identify key themes and terms.
    \item Use reader tool to read abstracts and determine relevance.
\end{enumerate}
\subsection{Extract information}
As a research student, I should be able to:
\begin{enumerate}
    \item Open Zotero application on computer.
    \item Use contexts tool on Voyant to connect to Zotero.
    \item Extract source and metadata from Voyant onto Zotero for storage.
   \end{enumerate}
\subsection{Store sources}
As a research student, I should be able to:
\begin{enumerate}
    \item Check that the extracted source appears in Zotero.
    \item Click on each source to ensure metadata is available.
    \item Click on each source to make sure the original website or pdf can be accessed.
    \item Flag source if metadata or part of metadata is missing, or if the source is not available.
\end{enumerate}
\subsection{Annotations}
As a research student, I should be able to:
\begin{enumerate}
    \item Add annotations to sources in Zotero.
    \item Access annotations in future, and add to them more.
    \end{enumerate}
\subsection{Grouping}
As a research student, I should be able to:
\begin{enumerate}
    \item Add tags or labels to source according to relevance or specific theme.
    \item Search for specific groups of sources by these tags.
\end{enumerate}
\subsection{Export metadata}
As a research student, I should be able to:
\begin{enumerate}
    \item Export source metadata from Zotero to Microsoft Word.
    \item Check that metadata is correct.
    \item Add to bibliography list if required.
\end{enumerate}

\subsection{Pre-requisites:}
\begin{itemize}
    \item All user stories must be completed in order for full efficiency. 
    \item An appropriate database must be selected for initial research phase to find sources. \textit{Macquarie University database was tested and selected during Elaboration}
    \item Voyant must have been tested prior with known sources to determine whether this is reliable. \textit{This was completed in elaboration.}
    \item Zotero must be installed on computer and a connecting icon added to the browser of choice (Mozilla Firefox) before beginning. \textit{This was completed in elaboration.}
    \end{itemize}
    
\section{Quality Assurance}
\subsection{Identify Commonalities:}
For Voyant to be successful, it must identify commonalities between the different sources. At least 3 key terms will be identified and will be reflected in the trends tool. It must also show where in the source is each term used.

\subsection{Extract Information:}
For Zotero to be successful, it must be able to extract both the source and the metadata from Voyant in one process. There must not be two separate processes for extracting the source and the metadata. 

\subsection{Store Sources:}
For Zotero to be successful, the source and metadata should be stored within the program library. They should be able to be accessed at future points and the original source (whether it is a website, or a file) can be accessed.

\subsection{Annotation:}
For Zotero to be successful, annotations should be able to be added to specific sources. These annotations should be saved and can be edited, or updated at future points of time. 

\subsection{Grouping:}
For Zotero to be successful, tags should be able to be added to sources depending on relevance, topic area or the needs of the researcher. A search of these tags in Zotero should produce all sources under this tag. 
\end{document}
